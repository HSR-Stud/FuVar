\section{Übungsverzeichnis}
\renewcommand{\arraystretch}{1.05}
\begin{tabular}{|l|l|l|l|}
\hline
\textbf{Thema} & \textbf{Übung} & \textbf{Probeprüfung} & \textbf{Modulprüfung 07} \\ \hline
Achsabschnitte der Tangentialebene & 2.2 &  &  \\ \hline
Approximation mit Tangentialebene & 2.3 &  &  \\ \hline
Bogenlängenparameter &  & 4 &  \\ \hline
Definitionsbereich & 1.2 &  &  \\ \hline
Differenzialgleichung &  &  & 3 \\ \hline
Differenzierbarkeit &  & 2 &  \\ \hline
Divergenz & 13.3 / 13.4 & 8 &  \\ \hline
Elektromagnetisches Feld & 13.4 &  &  \\ \hline
Extremalprobleme & 4.3 &  &  \\ \hline
Extremalstellen auf Randwerten & 5.1 / 5.2 &  &  \\ \hline
Falllinien & 4.2 & 9 &  \\ \hline
Feldstärke in Abhängigkeit vom Radius & 13.2 &  &  \\ \hline
Flächenänderung & 7.3 &  &  \\ \hline
Flächenberechnung mit der Greenschen Formel & 11.4 &  &  \\ \hline
Flugzeug am exponentiellen Berg & 3.5 &  &  \\ \hline
Fluss & 12.2 / 12.3 / 13.1 &  &  \\ \hline
Gauss (Satz) & 13.1 &  &  \\ \hline
Gebiete & 8.1 & 8 &  \\ \hline
Gradient & 3.1 &  &  \\ \hline
Greensche Formel & 11.2 / 11.3 &  & 8 \\ \hline
Grenzwerte eines Gradienten & 3.6 &  &  \\ \hline
Implizite Differentiation & 4.1 &  &  \\ \hline
Insektenmännchen & 3.4 &  &  \\ \hline
Jacobimatrix & 7.2 / 7.3 / 7.4 &  &  \\ \hline
Krümmung / Krümmungsradius & 5.3 & 4 &  \\ \hline
Krümmungmittelpunkt & 5.4 &  &  \\ \hline
Maximieren einer Funktion & 5.1 / 5.2 & 5 & 5 \\ \hline
Maximum & 4.3 & 1 &  \\ \hline
Mehrfachintegrale & 8.2 / 8.3 / 8.4 & 3 & 1 / 4 / 7 \\ \hline
Minimum & 4.3 & 1 &  \\ \hline
Newtonverfahren & 7.1 / 7.2 / 7.5 &  &  \\ \hline
Niveaulinien & 3.3 &  &  \\ \hline
Oberflächenberechnung & 9.3 / 10.3 &  &  \\ \hline
Parameterdarstellung &  & 3 & 1 \\ \hline
Partielle Ableitung & 2.1 &  &  \\ \hline
Pfadintegral, Wegintegral & 6 / 11.2 & 7 & 8 \\ \hline
Potentialfeld & 6 / 11.1 &  & 6 \\ \hline
Richtungsableitung &  & 2 &  \\ \hline
Rotation & 11.1 &  &  \\ \hline
Rotationskörper & 9.1 / 9.2 &  &  \\ \hline
Sattelpunkt & 4.3 & 1 &  \\ \hline
Schnittkurve von zwei Funktionen & 2.4 &  &  \\ \hline
Schwerpunkt berechnen & 10.1 / 10.2 / 10.3 / 10.4 &  &  \\ \hline
Steigung impliziter Funktionen & 4.1 &  &  \\ \hline
Stokes (Satz) & 12.1 &  & 6 \\ \hline
Tangentialebene & 2.2 / 2.3 &  &  \\ \hline
Trägheitsmoment & 10.2 & 6 &  \\ \hline
Volumenänderung & 7.4 &  &  \\ \hline
Volumenberechnung & 8.2 / 8.3 / 8.4 / 9.1 / 9.2 & 3 & 1 / 4 / 7 \\ \hline
Wertebereich & 1.2 &  &  \\ \hline
Winkel zwischen zwei Funktionen & 2.4 &  &  \\ \hline
\end{tabular}

\renewcommand{\arraystretch}{1}