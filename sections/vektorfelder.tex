\section{Vektorfelder\formelbuch{82}}
Ein Vektorfeld ist eine Abbildung $\vec{v}:\mathbb{R}^n\rightarrow
\mathbb{R}^n$, welche jedem Punkt des Raumes einen Vektor anheftet.
\subsection{Wegintegral\formelbuch{83}}
Ist $\vec{F}$ ein Vektorfeld und $C:\vec{c}=\vec{c}(t),a\leq t \leq b$ eine
Kurve, dann wird das Wegintegral wie folgt berechnet:\\\\
$\boxed{\int\limits_C\vec{F}\cdot d\vec{s}=
\int\limits_a^b\vec{F}(\vec{c}(t))\cdot \dot{\vec{c}}(t)\cdot dt}$\\\\
Tipp: Wenn c bspw. eine gerade Linie vom Punkt (-1;0) bis (1;0) ist, dann ist:\\
$\vec{c}(t)=
  \begin{pmatrix}
      x(t)\\
      y(t)\\
  \end{pmatrix}
  =
    \begin{pmatrix}
      t\\
      0\\
  \end{pmatrix}$
  mit $-1\leq t \leq 1$

\textbf{Eigenschaften des Wegintegrals}
\begin{itemize}
  \item Ist $C$ eine geschlossene Kurve, also $\vec{c}(a)=\vec{c}(b)$, spricht
  man von einem \textbf{Umlaufintegral} und verwendet folgende Definition:
  $\oint\limits_C \vec{F}\cdot d\vec{s}$
  \item Ist $C=C_1+C_2$ gilt folgendes:$\quad\int\limits_C \vec{F}\cdot
  d\vec{s}=\int\limits_{C_1} \vec{F}\cdot d\vec{s}+\int\limits_{C_2}
  \vec{F}\cdot d\vec{s}$
  \item Wird $C$ in Gegenrichtung durchlaufen gilt folgendes:
  $\quad\int\limits_C \vec{F}\cdot
  d\vec{s}=-\int\limits_{-C} \vec{F}\cdot d\vec{s}$
\end{itemize}

\subsection{Konservative Felder / Potentialfelder\formelbuch{83}}
Ein Vektorfeld $\vec{F}$ heisst \textit{konservativ}, wenn das Wegintegral
unabhängig vom gewählten Weg ist.\\\\
$\Rightarrow\qquad \boxed{\oint_C \vec{F}\cdot d\vec{s}=0}\qquad\boxed{
\int\limits_{Weg_1\, A\rightarrow B} \vec{F}\cdot d\vec{s}=\int\limits_{Weg_2\,
A\leadsto B} \vec{F}\cdot d\vec{s}=\int\limits_A^B \vec{F}\cdot
d\vec{s}}\qquad\boxed{\rotation (\vec{F})=0}$\\\\\\	
\textbf{Es gelten folgende Sätze:}
Für ein Vektorfeld $\vec{F}$ im zwei- oder dreidimensionalen Koordinatensystem
gilt in einem Bereich ohne "`durchgehende Löcher"', d.h. in dem sich jede
geschlossene Kurve auf einen Punkt zusammenziehen lässt:
\begin{itemize}
  \item $\vec{F}$ ist konservativ, d.h die Kurvenintegrale wegunabhängig.
  \item $\vec{F}$ ist ein Gradientenfeld, d.h. es gibt ein Potential $\Phi$ mit
  $\vec{F}=\gradient \Phi(\vec{x}(t))$.
  \item $\vec{F}$ erfüllt die sogenannte(n)
  \textbf{Integrabilitätsbedingung(en)}, d.h. im
  \begin{itemize}
    \item zweidimensionalen Fall die Gleichung $\frac{\partial F_1}{\partial
    x_2}=\frac{\partial F_2}{\partial x_1}$,
    \item dreidimensionalen Fall die Gleichungen $\frac{\partial F_1}{\partial
    x_2}=\frac{\partial F_2}{\partial x_1}$, $\frac{\partial F_2}{\partial
    x_3}=\frac{\partial F_3}{\partial x_2}$, $\frac{\partial F_3}{\partial
    x_1}=\frac{\partial F_1}{\partial x_3}$
  \end{itemize}
\end{itemize}
\textbf{und}
\begin{itemize}
  \item Ist $\vec{F}=\gradient \Phi(\vec{x}(t))$, dann ist $\vec{F}$  
  \textit{konservativ} und es gilt: $\quad\int\limits_r^s f(t)
  dt=\Phi(s)-\Phi(r)$
  \item Ist $\vec{F}$ \textit{konservativ}, dann gibt es eine Funktion $\Phi$
  so, dass $\vec{F}=\gradient \Phi(\vec{x}(t))=
  \begin{pmatrix}
    \Phi_{x_1}\\
    \vdots\\ 
    \Phi_{x_n}
  \end{pmatrix}$.

\end{itemize}
\newpage
\subsection{Rotation\formelbuch{86 \& 92}}
Interpretiert man dieses Feld als Strömungsfeld, so gibt die Rotation für jeden
Ort an, wie schnell und um welche Achse ein mitschwimmender Körper rotieren
würde. Ein Vektorfeld, dessen Rotation überall null ist, nennt man
\textit{wirbelfrei}.\\\\
\begin{minipage}{8cm}
	\textbf{2-Dimensional}\\\\
	$\boxed{\rotation(u,v)=\fprt{v}{x}-\fprt{u}{y}}$
\end{minipage}
\begin{minipage}{8cm}
	\textbf{3-Dimensional}\\\\
	$\boxed{\rotation(u,v,w)=\bigtriangledown\times
	\vec{v}=
	\begin{pmatrix}
    	\partial_x\\
    	\partial_y\\
    	\partial_z
    \end{pmatrix}\times
	\begin{pmatrix}
    	v_x\\
    	v_y\\
    	v_z
    \end{pmatrix}=
	\begin{pmatrix}
    	\fsprt{y}{w}-\fsprt{z}{v}\\
    	\fsprt{z}{u}-\fsprt{x}{w}\\
    	\fsprt{x}{v}-\fsprt{y}{u}
    \end{pmatrix}}$
\end{minipage} \\

\subsection{Divergenz\formelbuch{96}}
Interpretiert man dieses Feld als Strömungsfeld, so gibt die Divergenz für jede
Stelle die Tendenz an, ob ein Teilchen in der Nähe zu diesem Punkt hin- bzw.
von diesem Punkt wegfließt. Es sagt damit aus, ob und wo das Vektorfeld Quellen
(Divergenz größer als Null) oder Senken (Divergenz kleiner als Null) hat. Ist
die Divergenz überall gleich Null, so bezeichnet man das Feld als
\textit{quellenfrei}.\\\\
$\boxed{\divergenz \vec{v}=\bigtriangledown\circ \vec{v}=
	\begin{pmatrix}
    	\partial_{x_1}\\
    	\vdots\\
    	\partial_{x_n}
    \end{pmatrix}\cdot
	\begin{pmatrix}
    	v_{x_1}\\
    	\vdots\\
    	v_{x_n}
    \end{pmatrix}=
\sum\limits_{i=1}^n \fprt{v_i}{x_i}}$

\subsection{Fluss\formelbuch{93}}
Der Fluss des Vektorfeldes $\vec{v}$ durch das Flächenstück G mit
Parametrisierung $f:U\rightarrow\mathbb{R}^3$ ist:\\\\
$\boxed{\iint\limits_G\vec{v}\cdot d\vec{n}=\iint\limits_U \vec{v}\cdot
(\sprt{u}\times\sprt{v}) dudv}$

\subsection{Integralsätze}
	\begin{tabular}{|p{0.4cm}||p{4cm}|p{5.7cm}|p{7cm}|}
	\hline
	& \textbf{Satz} & \textbf{Formel} & \textbf{Beschreibung}\\
	\hline
	\hline
	\begin{sideways}2-D \qquad \end{sideways} &
  	Die Greensche Formel \formelbuch{88} &
	\begin{minipage}{6.7cm}
	    \vspace{0.1cm}
		$\int\limits_{\partial G}\vec{v}\cdot d\vec{s}=\iint\limits_G \rotation
		\vec{v}\cdot d\mu$	 		    
	    \vspace{0.1cm}   
    \end{minipage}&
	\begin{minipage}{7cm}
	    \vspace{0.1cm}
		Die Greensche Formel beschreibt den Zusammenhang zwischen einem
		\textbf{Wegintegral} und einem \textbf{Oberflächenintegral}.
	    \vspace{0.1cm}   
    \end{minipage}\\
	\hline
	\begin{sideways}3-D \qquad \end{sideways} &
	Der Satz von Stokes \formelbuch{94} &
	\begin{minipage}{6.7cm}
    	\vspace{0.1cm}
		$\int\limits_{\partial G}\vec{v}\cdot d\vec{s}=\iint\limits_G \rotation
		\vec{v}\cdot d\vec{n}$		 
		\vspace{0.1cm} 
    \end{minipage}&
	\begin{minipage}{7cm}
    	\vspace{0.1cm}
		Der Satz von Stokes definiert einen Zusammenhang zwischen einem
		\textbf{Wegintegral} und einem \textbf{Flussintegral}.	    
	    \vspace{0.1cm}   
    \end{minipage}\\
	\hline
	\begin{sideways}3-D \qquad \end{sideways} &
	Der Satz von Gauss \formelbuch{96} &
	\begin{minipage}{6.7cm}
	    \vspace{0.1cm}
		$\iint\limits_{\partial V}\vec{v}\cdot d\vec{n}=\iiint\limits_V \divergenz
		\vec{v}\cdot dV$			    
	    \vspace{0.1cm}   
    \end{minipage}&
	\begin{minipage}{7cm}
    	\vspace{0.1cm}
		Der Satz von Gauss definiert einen Zusammenhang zwischen einem
		\textbf{Flussintegral} und einem \textbf{Volumenintegral}.	    
	    \vspace{0.1cm}    
    \end{minipage}\\
	\hline	
\end{tabular}
