\section{Skalare Funktionen ($\mathbb{R}^n \rightarrow \mathbb{R}$)
\formelbuch{17}}

\subsection{Ableitungen\formelbuch{21}}
\begin{tabular}{|l|l|l|}
\hline
\textbf{partielle Ableitung} & \textbf{Gradient} & \textbf{Richtungsableitung}\\
\hline
\begin{minipage}{5.5cm}
	\vspace{0.2cm}
	$\dprt{x_1}(x_1,\ldots,x_n)= \sprt{x_1}(x_1,\ldots,x_n)$
	\vspace{0.2cm}
\end{minipage}&
\begin{minipage}{6cm}
	$\gradient f(x_1,\ldots,x_n)= \left(\dprt{x_1}, \,\ldots\,,
	\dprt{x_n}\right)$ \end{minipage}&
\begin{minipage}{6.5cm}
	\vspace{0.1cm}
	 Allgemein:\\\vspace{0.1cm}
  $\dprt{\vec{r}}(x_1, \ldots, x_n)=\gradient f(x_1, \ldots, x_n)
  \bullet \vec{r}$\\
  \vspace{0.3cm}
	an der Stelle $(x_0;y_0)$:\\\vspace{0.1cm}
	$\dprt{\vec{r}}(x_0;y_0)=\gradient f(x_0;y_0) \bullet \vec{r}$\\

	
	\vspace{0.1cm}
\end{minipage}\\
\hline
\begin{minipage}{5.5cm}
	\vspace{0.2cm}
	partielle Ableitung der Funktion $f$ nach dem Parameter $x_1$
	\vspace{0.2cm}
\end{minipage}&
\begin{minipage}{5.5cm}
	Vektor der partiellen Ableitungen von $f$
\end{minipage}&
\begin{minipage}{5.5cm}
	Steigung der Funktion $f$ in Richtung des Vektors $\vec{r}\quad (|\vec{r}|=1)$
\end{minipage}\\
\hline
\end{tabular}

\subsection{Tangentialebene\formelbuch{12}}
$\boxed{g(x,y)=f(x_0,y_0)+\dprt{x}(x_0,y_0)\cdot(x-x_0)+\dprt{y}(x_0,y_0)\cdot(y-y_0)}
\quad \Rightarrow \quad$ Tangentialebene im Punkt $(x_0,y_0,f(x_0,y_0))$\\ \\
\textbf{Für Approximation einer Formel}\\
Linearisierung am Punkt $(x_0,y_0)$: Tangentialebene $g(x;y)$ beim Punkt
$(x_0;y_0)$\\
Die Lösung des approximierten Wertes liefert dann $g(x;y)$.\\

$\boxed{\vec{n}(x_0,y_0)=
\begin{pmatrix}
	\sprt{x}(x_0,y_0)\\
	\sprt{y}(x_0,y_0)\\
	-1\\                         
\end{pmatrix}} \quad \Rightarrow \quad$ Normalenvektor der Tangentialebene

\subsection{Niveaulinien und Falllinien\formelbuch{14}}
\textbf{Falllinien, Orthogonaltrajektorien}\\
Falllinien sind Kurven, die in jedem Punkt in Richtung max. Zunahme von f
verlaufen\\
$\Rightarrow \gradient f$ steht tangential zur Falllinie\\\\
$\boxed{y'(x)= \dfrac{\prt{y}}{\prt{x}}} \quad
\Rightarrow \quad$ Steigung der Falllinie $y(x)$\\\\
Durch lösen dieser Differentialgleichung erhält man
die Lösungskurve $y(x)$\\ \\
\textbf{Niveaulinien}\\
$\gradient f \perp $ Niveaulinie oder Tangentensteigung\\\\
$\boxed{y'(x)=- \dfrac{\prt{x}}{\prt{y}}}\quad
\Rightarrow \quad$ Steigung der Niveaulinie $y(x)$\\\\
Durch lösen dieser Differentialgleichung erhält man
die Lösungskurve $y(x)$. Auf diese Weise kann man auch einfach die
Tangentensteigung einer impliziten Form z.B. $\frac{x^2}{100}+\frac{y^2}{25}=1$
lösen

\subsection{Das totale Differenzial\formelbuch{19}}
$\boxed{\Delta z \approx dz=f_x(x_0;y_0)dx+f_y(x_0;y_0)dy}$\\
$\boxed{z=\bar z \pm \sigma_{\bar z} \approx f(\bar x, \bar y) \pm
\sqrt{(f_x(\bar x,\bar y)\cdot \sigma_{\bar x})^2+(f_y(\bar x,\bar y)\cdot
\sigma_{\bar y})^2}}$

\subsection{Zusammengesetzte Funktionen\formelbuch{34}}
$\boxed{\gradient(g\circ f)(x)=g'(f(x))\cdot \gradient f(x)}$
\newpage

\subsection{Extremalprobleme}
\textbf{Zwei Dimensionen}\\
\begin{enumerate}
  \item Randpunkte von $\mathbb{D}_f$
  \item Punkte, in denen der Gradientenvektor $\gradient f$ nicht existiert. 
  \item Punkte, in denen der Gradientenvektor $\gradient f=0$ ist.\\
  Hat die Funktion $f(x;y)$ an der Stelle $(x_0;y_0)$ einen verschwindenden Gradientenvektor $\gradient f = 0$ und gilt für die Diskriminante $\Delta = f_{xx}    (x_0;y_0) \cdot f_{yy}(x_0;y_0) -  (f_{xy}(x_0;y_0))^2$
  \begin{itemize}
    \item $\Delta > 0 $, so besitz $f(x_0;y_0)$ ein lokales Extremum. Im Fall $f_{xx}(x_0;y_0) < 0$ liegt ein lokales Maximum vor, für $f_{xx}(x_0;y_0) >       0$ hingegen ein lokales Minimum.
    \item $\Delta < 0 $, so besitzt $f(x_0;y_0)$ in $(x_0;y_0)$ ein Sattelpunkt.
    \item $\Delta = 0 $, so braucht es weitere Untersuchungen, um die Art der Stelle $(x_0;y_0)$ zu bestimmen können.
    \item Wenn es kein lokales/relatives Minima/Maxima gibt, dann auch kein
    absolutes! 
  \end{itemize}
\end{enumerate}
\textbf{m Dimensionen}\\
Funktion $f(x_1^{(0)};\ldots;x_m^{(0)})$ gegeben.  \\
\begin{itemize}
  \item Schritte 1 - 3 sind gleich wie bei zwei Dimensionen. Kandidaten $(x_1^{(0)};\ldots;x_m^{(0)})$ bekommen.
  \item Bestimmung der Art der Extremalstellen: Hessesche Matrix aufstellen\\
  $H(x_1^{(0)};\ldots;x_m^{(0)}) := \begin{pmatrix}
          f_{x_{1}x_{1}}(x_1^{(0)};\ldots;x_m^{(0)}) & \ldots & f_{x_{1}x_{m}}(x_1^{(0)};\ldots;x_m^{(0)})\\
          \vdots && \vdots\\
          f_{x_{m}x_{1}}(x_1^{(0)};\ldots;x_m^{(0)}) & \ldots & f_{x_{m}x_{m}}(x_1^{(0)};\ldots;x_m^{(0)})
          \end{pmatrix}$\\
  \begin{itemize}
    \item $m$ positive Eigenwerte $\lambda_i > 0$, so besitzt $f$ in $(x_1^{(0)};\ldots;x_m^{(0)})$ ein lokales Minimum,
    \item $m$ negative Eigenwerte $\lambda_i < 0$, so besitzt $f$ in $(x_1^{(0)};\ldots;x_m^{(0)})$ ein lokales Maximum,
    \item positive und negative Eigenwerte, so besitzt $f$ in $(x_1^{(0)};\ldots;x_m^{(0)})$ einen Sattelpunkt.
    \item Wenn $\lambda_i \leqq 0$ oder $\lambda_i \geqq 0$ sind weitere Untersuchungen nötig.
  \end{itemize}
\end{itemize}

\subsubsection{Extremstellen auf Randwerten \formelbuch{38}}
\begin{tabular}{lll}
	\begin{minipage}{3.5cm}
		$\boxed{\gradient f - \lambda \gradient g = 0}$		
	\end{minipage} &
	\begin{minipage}{4.7cm}
		$f: $ zu maximierende Funktion\\
		$g: $ Funktion des Randes	
    \end{minipage} &
	\begin{minipage}{11cm}
		1. Gleichung unter der Bedingung  $g=0$ auflösen\\
		2. Lösungen durch einsetzen in $f$ auf Min o. Max untersuchen
    \end{minipage}
\end{tabular}


\subsection{Extremalprobleme mit Nebenbedingungen}
\textbf{Zwei Dimensionen}\\
Gegeben: $f(x;y)$ unter der Nebenbedingung $n(x;y) = 0$\\
So kommen folgende Punkte von $f$ in Frage:
\begin{enumerate}
  \item Randpunkte von $\mathbb{D}_f$ wenn sie die Nebenbedingungen  $n(x;y) = 0$ erfüllen.
  \item Punkte, in denen der Gradientenvektor $\gradient f $ und / oder $\gradient n$ nicht existiert, und die Nebenbedingung nicht erfüllt ist,
  \item Lösungen des Gleichungsystems\\
  $ \begin{vmatrix}
      f_x(x;y) \cdot n_y(x;y) = f_y(x;y) \cdot n_x(x;y) \\
      n(x;y) = 0
    \end{vmatrix} $
   \item Lösung in $f(x,y)$ einsetzen und untersuchen!
\end{enumerate}
