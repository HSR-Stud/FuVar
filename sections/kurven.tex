\section{Kurven ($\mathbb{R} \rightarrow \mathbb{R}^n$)\formelbuch{41}}

%\subsection{Parametertransformation\formelbuch{42}}
%Tangentenrichtung ist unabhängig von der Parametrisierung der Kurve. Die Länge
%jedoch nicht. Ist $g$ eine Kurve, beschreibt $f\circ g$ die gleiche Kurve mit
%einer anderen Parametrisierung. Der Tangentialvektor ist somit:\\
%$$\frac{d(f\circ g)}{dt}(t_0)=\frac{df}{dt}(g(t_0))\frac{dg}{dt}(t_0)=
%\frac{df}{dt}(g(t_0))g'(t_0)$$\\
%die Länge des Vektors wird also mit dem Faktor $f'(t_0)$ gestreckt oder
%gestaucht.

\subsection{Flächeninhalt einer geschlossenen ebenen Kurve\formelbuch{73}}
  Eine kreuzungsfreie, geschlossene Kurve $\begin{pmatrix} x\\ y \end{pmatrix} = \begin{pmatrix} x(t)\\ y(t) \end{pmatrix}, t \in (r;s)$ 
  hat einen Flächeninhalt von\\
  $\boxed{F =  \left | \int_r^s y(t) \cdot x^{'}(t) dt \right | = \left | \int_r^s
  x(t) \cdot y^{'}(t) dt \right |}$\\\\
  Hat die Kurve einen Kreuzungspunkt:\\ 
  $x_1(t_1) = x_2(t_2)$\\
  $y_1(t_1) = y_2(t_2)$\\\\
  Der Zeitpunkt t ist bei $x_1$ und $x_2$ \textbf{NICHT} gleich!\\
  So kann man herausfinden, innerhalb welchen Zeitpunkten die Schleife durchgegangen wird!\\

\begin{multicols}{2}  
  \subsection{Differentation einer Kurve\formelbuch{75}}
    $\boxed{\begin{pmatrix}
      f_1(t) \\
      \vdots \\
      f_m(t)
    \end{pmatrix}' = \begin{pmatrix}
      f_1'(t)\\
      \vdots\\
      f_m'(t)
    \end{pmatrix}}$ \\
    $\Rightarrow$ Tangente nicht partiell ableiten sondern nach $t$
  \columnbreak  
  
  \subsection{Kurvenintegrale\formelbuch{77}}
    $\boxed{I = \int\limits_r^s f(\vec x(t)) \cdot |\vec x'(t)| dt}$
\end{multicols}

\subsection{Kurvenlänge\formelbuch{79}}
\begin{tabular}{|l||l|l|}
\hline
& \textbf{Formel} & \textbf{Bedingung}\\
\hline
\hline
\textbf{Parametergleichung $\vec{x}=\vec{x}(t)$} &
	\begin{minipage}{6cm}
    	\vspace{0.1cm}
		$l=\int\limits_r^s\sqrt{[x_1'(t)]^2+\ldots+[x_n'(t)]^2}dt$ 
		\vspace{0.1cm}
    \end{minipage}&
  $r\leq t\leq s$\\
\hline
\textbf{Im Speziellen für Graphen von Funktionen $f(x)$} &
	\begin{minipage}{6cm}
    	\vspace{0.1cm}
		$l=\int\limits_a^b\sqrt{1+[f'(x)]^2}dx$ 
		\vspace{0.1cm}
    \end{minipage}&
  $a\leq x \leq b$\\
\hline
\end{tabular}\vspace{0.5cm}\\





%\subsection{Krümmung \& Krümmungskreisradius\formelbuch{45}}
%\begin{tabular}{|l|l|l|}
%	\hline
%	& mit Kurvenlägenparameter & ohne Kurvenlängenparameter\\
%	\hline
%	Krümmung &
%	$\kappa(s)=|f''(s)|$ &
%	\begin{minipage}{5cm}
%    	\vspace{0.1cm}
%		$\kappa(t)=\dfrac{|f'(t) \times f''(t)|}{|f'(t)|^3}$\\ 
%		\vspace{0.1cm}
%    \end{minipage}\\	
%	\hline
%	Krümmungskreisradius &
%	$r(s)=\dfrac{1}{\kappa(s)}=\dfrac{1}{|f''(s)|}$ &
%	\begin{minipage}{5cm}
%    	\vspace{0.1cm}
%		$r(t)=\dfrac{1}{\kappa(t)}=\dfrac{|f'(t)|^3}{|f'(t) \times
%		f''(t)|}$    
%		\vspace{0.1cm}
%    \end{minipage}\\
%	\hline
%\end{tabular}
%\newpage
