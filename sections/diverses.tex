\section{Diverses}

\subsection{Rechenregeln für div, rot, grad, $\Delta$}
\begin{minipage}{9cm}
Sei $D \in \mathbb{R}^3, f \in C^2(D,\mathbb{R}), \vec{v} \in C^2(D,\mathbb{R}^3)$. Dann gilt:
\begin{itemize}
	\item $\rotation(\gradient f) = 0 \qquad $ ``Gradientenfeld ist wirbelfrei''
	\item $\divergenz(\rotation \vec{v}) = 0 \qquad $ ``Feld der Rotation ist quellfrei''
	\item $\divergenz(\gradient f) = \Delta f$
	\item $\divergenz(f\vec{v}) = (\gradient f) \cdot \vec{v} + f \divergenz \vec{v}$
	\item $\rotation(f\vec{v}) = (\gradient f) \times \vec{v} + f \rotation \vec{v}$
	\item $\rotation(\rotation \vec{v}) = \gradient(\divergenz \vec{v}) - \Delta \vec{v}$
\end{itemize}
(Laplace–Operator kompomentenweise anwenden!) .
\end{minipage}
\hspace{1cm}
\begin{minipage}[b]{6cm}
\textbf{Laplace-Operator($\Delta$):}\\
$\Delta=\vec\nabla^2= \sum_{k=1}^n {\partial^2\over \partial x_k^2}$\\\\
\textbf{ Nabla-Operator($\nabla$):}\\
$\vec\nabla = \left (\frac\partial{\partial x_1},\ldots, \frac\partial{\partial
x_n}\right) $
\end{minipage}

\subsection{Eigenwerte}
  Die Eigenwerte $\lambda$ erhält man folgendermassen ($I$ ist die Einheitsmatrix):\\
  $|\lambda I - A| = 0 \qquad \Rightarrow$ nach $\lambda$ auflösen
  
  
\subsection{Determinante}
  \[ det \begin{pmatrix}
    a & b \\
    c & d
  \end{pmatrix} = ad - bc \\
  det \begin{pmatrix}
    a & b & c \\
    d & e & f \\
    g & h & i
  \end{pmatrix} = aei + bfg + cdh - ceg - afh - bdi \]