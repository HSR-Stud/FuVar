%%%%%%%%%%%%%%%%%%%%%%%%%
% Dokumentinformationen %
%%%%%%%%%%%%%%%%%%%%%%%%%
\newcommand{\titleinfo}{Funktionen mehrerer Variablen - Formelsammlung}
\newcommand{\authorinfo}{F. Braun, die Schweisser, C.Gwerder, S.Koerner}
\newcommand{\versioninfo}{$Revision: 2.0$ - gem\"ass Unterricht Remo Bernhardsgr\"utter/FS2013}

%%%%%%%%%%%%%%%%%%%%%%%%%%%%%%%%%%%%%%%%%%%%%
% Standard projektübergreifender Header für
% - Makros 
% - Farben
% - Mathematische Operatoren 
%
% DORT NUR ERGÄNZEN, NICHTS LÖSCHEN
%%%%%%%%%%%%%%%%%%%%%%%%%%%%%%%%%%%%%%%%%%%%%  
%Schriftgr"osse, Layout, Papierformat, Art des Dokumentes
\documentclass[10pt,a4paper,fleqn,headsepline,footsepline]{scrartcl}
%Einstellungen der Seitenränder
\usepackage[left=0.8cm,right=0.8cm,top=0.5cm,bottom=0.5cm,includeheadfoot]{geometry}
% Sprache, Zeichensatz, packages
\usepackage[UTF8]{inputenc}
\usepackage[ngerman]{babel,varioref}
\usepackage{amssymb,amsmath,graphicx,xcolor,lastpage,wrapfig,verbatim}
\usepackage{tabularx,longtable}
\usepackage{multicol}
\usepackage{rotating}
\usepackage{floatflt}
\usepackage{array}
\usepackage{scrlayer-scrpage}
\usepackage{multirow,multicol}
\usepackage{trfsigns, trsym}
\usepackage{tikz}
\usepackage{circuitikz}
\usepackage{afterpage}

% Zum Bilder einfach in Tabellen einfügen (valign=t)
\usepackage[export]{adjustbox}
%
\setkomafont{pageheadfoot}{\footnotesize}
%
%
\RedeclareSectionCommands[
  beforeskip=.2\baselineskip,
  afterskip=.2\baselineskip
]{section,subsection,subsubsection,paragraph}

\definecolor{pgrey}{rgb}{0.2,0.2,0.2}
\definecolor{black}{rgb}{0,0,0}
\definecolor{red}{rgb}{1,0,0}
\definecolor{white}{rgb}{1,1,1}
\definecolor{grey}{rgb}{0.8,0.8,0.8}
\definecolor{green}{rgb}{0,.8,0.05}
\definecolor{brown}{rgb}{0.603,0,0}

\DeclareMathOperator{\sinc}{sinc}
\DeclareMathOperator{\sgn}{sgn}
\DeclareMathOperator{\Real}{Re}
\DeclareMathOperator{\Imag}{Im}
%\DeclareMathOperator{\e}{e}
\DeclareMathOperator{\cov}{cov}
\DeclareMathOperator{\PolyGrad}{PolyGrad}

\newcommand{\HRule}{\noindent\rule{\linewidth}{1pt}}
%
\newcommand{\myparagraph}[1]{\paragraph{#1}\mbox{}\\\nopagebreak}
\newcommand{\formelbuch}[1]{$\quad{\textcolor{pgrey}{\mbox{\small{S#1}}}}$}
\newcommand{\hartl}[1]{$\quad{\textcolor{pgrey}{\mbox{\small{S#1}}}}$}


\newcommand*{\diff}{\mathop{}\!\mathrm{d}}
\newcommand{\FT}
{
\begin{picture}(1,0.5)
\put(0.2,0.1){\circle{0.14}}\put(0.27,0.1){\line(1,0){0.5}}\put(0.77,0.1){\circle*{0.14}}
\end{picture}
}

\newcommand{\IFT}
{
\begin{picture}(1,0.5)
\put(0.2,0.1){\circle*{0.14}}\put(0.27,0.1){\line(1,0){0.45}}\put(0.77,0.1){\circle{0.14}}
\end{picture}
}


\newcommand{\arraystretchOriginal}{1.3} %%1.5
\renewcommand{\arraystretch}{\arraystretchOriginal}


\newcolumntype{L}[1]{>{\raggedright\let\newline\\\arraybackslash\hspace{0pt}}m{#1}}
\newcolumntype{C}[1]{>{\centering\let\newline\\\arraybackslash\hspace{0pt}}m{#1}}
\newcolumntype{R}[1]{>{\raggedleft\let\newline\\\arraybackslash\hspace{0pt}}m{#1}}


\author{\authorinfo}
\title{\titleinfo}
%
%Kopf- und Fusszeile
%
\lohead*{\titleinfo}
\rohead*{\today}
\lofoot*{\authorinfo}
\cofoot*{\includegraphics[width=1.6cm]{header/small.png}}
\rofoot*{Seite \thepage { }von \pageref{LastPage}}
%
\pagestyle{scrheadings}




%%%%%%%%%%%%%%%%%%%%
% Generelle Makros %
%%%%%%%%%%%%%%%%%%%%
\newcommand{\prt}[1]{\frac{\partial f}{\partial #1}}
\newcommand{\dprt}[1]{\dfrac{\partial f}{\partial #1}}
\newcommand{\sprt}[1]{\partial_{#1} f}
\newcommand{\fprt}[2]{\dfrac{\partial #1}{\partial #2}}
\newcommand{\fsprt}[2]{\partial_{#1} #2}

%%%%%%%%%%%%%%%%%%%%%%%%%%%%
% Mathematische Operatoren %
%%%%%%%%%%%%%%%%%%%%%%%%%%%%
\DeclareMathOperator{\gradient}{grad}
\DeclareMathOperator{\rotation}{rot}
\DeclareMathOperator{\divergenz}{div}

% Möglichst keine Ergänzungen hier, sondern in header.tex
\begin{document} 
 

%%%%%%%%%%%%%%%%%%%%%%%%%%%%%%%%%%%%%%%%%%%%%%%%%%%%%%%%%%%%%%%%%%%%%%%%%%%%%%%%%%%%%%%%%%%%%%%
%%%%%%%%%%%%%%%%%%%%%%%%%%%%%%%%%%%%%%%%%%%%%%%%%%%%%%%%%%%%%%%%%%%%%%%%%%%%%%%%%%%%%%%%%%%%%%%

\section{Skalare Funktionen ($\mathbb{R}^n \rightarrow \mathbb{R}$)}

\subsection{Ableitungen}
\begin{tabular}{|p{5.3cm}|p{6.5cm}|p{6.5cm}|}
  \hline
    \textbf{partielle Ableitung\formelbuch{11/28}} & \textbf{Gradient\formelbuch{22/46}} & \textbf{Richtungsableitung\formelbuch{20/46}}\\
  \hline
    $\dprt{x}(x_0;y_0) := f_x(x_0;y_0)$ &
    
    $\gradient f(x_0,y_0)= \begin{pmatrix}
      f_x(x_0;y_0) \\ 
      f_y(x_0;y_0)
    \end{pmatrix} $ &
    
    An der Stelle $(x_0;y_0)$ \newline    
    $\dprt{\vec{r}}(x_0;y_0)=\gradient f(x_0;y_0) \bullet \vec{r}$ \\
  \hline
    $\fprt{^2f}{x^2}(x;y) = f_{xx}(x;y)$ \newline
    $\fprt{^2f}{x\partial y}(x;y) = f_{xy}(x;y) = f_{yx}(x;y)$ \newline
    $\fprt{^2f}{y^2}(x;y) = f_{yy}(x;y)$ &
    
    $\gradient f(x_1; \ldots;x_m) = \begin{pmatrix}
      f_{x_1}(x_1; \ldots;x_m)\\
      \vdots \\
      f_{x_m}(x_1; \ldots;x_m)
    \end{pmatrix}$ & 
    
    $\dprt{\vec{r}}(x_1, \ldots, x_m)=\gradient f(x_1, \ldots, x_m)
    \bullet \vec{r}$ \\
  \hline
    partielle Ableitung der Funktion $f$ nach dem Parameter $x_1$ &
    
    Vektor der partiellen Ableitungen von $f$. Der Gradient zeigt in Richtung der grössten Steigung und steht senkrecht auf der Niveaulinie.
    	Der Wert der Steigung entspricht $|\gradient f|$ &
    	
    	Steigung der Funktion $f$ in Richtung des Vektors 
    	$\vec{r} = \begin{pmatrix}
    	  \cos\alpha \\
    	  \sin\alpha
    	\end{pmatrix} (|\vec{r}|=1)$ \\
  \hline
\end{tabular}


\subsection{Tangentialebene}
  \begin{tabular}{|p{9cm}|p{9cm}|}
    \hline
      \textbf{2 Variablen\formelbuch{12}} &
      \textbf{m Variablen\formelbuch{44}} \\

      im Punkt: $x_0; y_0$ &
      im Punkt: $x_1^{(0)};\, \ldots \,; x_m^{(0)}$ \\
        
    \hline
      \[g(x,y)=f(x_0;y_0)+f_x(x_0;y_0)\cdot(x-x_0)+f_y(x_0;y_0)\cdot(y-y_0)\] &
      \begin{eqnarray}
        g(x_1;\ldots;x_m)  & = & f(x_1^{(0)};\ldots;x_m^{(0)}) + f_{x_1}(x_1^{(0)};\ldots;x_m^{(0)}) \cdot \nonumber \\
        & & (x_1-x_1^{(0)})+\ldots + f_{x_m}(x_1^{(0)};\ldots;x_m^{(0)}) \cdot \nonumber \\
        & & (x_m-x_m^{(0)}) \nonumber
      \end{eqnarray} \\
    \hline  
  \end{tabular}
  
\textbf{Für Approximation einer Formel}\\
Linearisierung am Punkt $(x_0;y_0)$: Tangentialebene $g(x;y)$ beim Punkt
$(x_0;y_0)$\\
Die Lösung des approximierten Wertes liefert dann $g(x;y)$.\\

$\boxed{\vec{n}(x_0,y_0)=
\begin{pmatrix}
	f_x(x_0,y_0)\\
	f_y(x_0,y_0)\\
	-1\\                         
\end{pmatrix}} \quad \Rightarrow \quad$ Normalenvektor der Tangentialebene

\subsection{Steigung\formelbuch{14}}
  $\boxed{m = -\frac{f_x(x_0;y_0)}{f_y(x_0;y_0)}}$ falls in einer impliziten Form $f(x;y) = 0$ und $f_y(x_0;y_0) \neq 0$
  

%\subsection{Niveaulinien und Falllinien\formelbuch{14}}
%\textbf{Falllinien, Orthogonaltrajektorien}\\
%Falllinien sind Kurven, die in jedem Punkt in Richtung max. Zunahme von f
%verlaufen\\
%$\Rightarrow \gradient f$ steht tangential zur Falllinie\\\\
%$\boxed{y'(x)= \dfrac{\prt{y}}{\prt{x}}} \quad
%\Rightarrow \quad$ Steigung der Falllinie $y(x)$\\\\
%Durch lösen dieser Differentialgleichung erhält man
%die Lösungskurve $y(x)$\\ \\
%\textbf{Niveaulinien}\\
%$\gradient f \perp $ Niveaulinie oder Tangentensteigung\\\\
%$\boxed{y'(x)=- \dfrac{\prt{x}}{\prt{y}}}\quad
%\Rightarrow \quad$ Steigung der Niveaulinie $y(x)$\\\\
%Durch lösen dieser Differentialgleichung erhält man
%die Lösungskurve $y(x)$. Auf diese Weise kann man auch einfach die
%Tangentensteigung einer impliziten Form z.B. $\frac{x^2}{100}+\frac{y^2}{25}=1$
%lösen

\subsection{Das totale Differenzial}

  \textbf{Die vier Grundoperationen} \\
  \begin{tabular}{|l|l|}
    \hline
      $z = f(x;y)$ &
      Standardabweichung \\
    \hline
      \begin{tabular}{l}
        $z = x + y$ \\
        $z = x - y$
      \end{tabular}  &    
      $\rbrace \sigma_{\bar z} \approx \sqrt{\sigma_{\bar x}^2 + \sigma_{\bar y}^2}$ \\
      
    \hline
      \begin{tabular}{l}
        $z = C\cdot x \cdot y$ \\
        $z = C\cdot \frac{x}{y}$
      \end{tabular} &
      $\rbrace |\frac{\sigma_{\bar z}}{\bar z}| \approx \sqrt{|\frac{\sigma_{\bar x}}{\bar x}|^2 + |\frac{\sigma_{\bar y}}{\bar y}|^2}$ \\
    \hline
  \end{tabular}
    

  \subsubsection{totale Differential}
    \begin{tabular}{|p{7cm}| l |p{10cm}| l}
      \cline{1-1}
      \cline{3-3}
        $\Delta z \approx dz=f_x(x_0;y_0)dx+f_y(x_0;y_0)dy$ &
        \formelbuch{16} &
        $\Delta f \approx df=f_{x_1}(x_1^{(0)};\ldots;x_m^{(0)})\cdot dx_1 + \ldots + f_{x_m}(x_1^{(0)};\ldots;x_m^{(0)})\cdot dx_m$ &
        \formelbuch{45} \\
      \cline{1-1}
      \cline{3-3}
    \end{tabular}      
  
  
  \subsubsection{Gauss'sches Fehlerfortpflanzungsgesetz}
    $\boxed{z=\bar z \pm \sigma_{\bar z} \approx f(\bar x, \bar y) \pm
    \sqrt{(f_x(\bar x,\bar y)\cdot \sigma_{\bar x})^2+(f_y(\bar x,\bar y)\cdot
    \sigma_{\bar y})^2}}$ \formelbuch{19} \\
    
    $\boxed{y=\bar y \pm \sigma_{\bar y} \approx f(\bar{x_1};\ldots;\bar{x_m}) \pm
    \sqrt{[f_{x_1}(\bar{x_1};\ldots;\bar{x_m}) \cdot \sigma_{\bar{x_1}}]^2 + \ldots + 
    [f_{x_m}(\bar{x_1};\ldots;\bar{x_m}) \cdot \sigma_{\bar{x_m}}]^2}}$ \formelbuch{45}
   

%\subsection{Zusammengesetzte Funktionen\formelbuch{34}}
%$\boxed{\gradient(g\circ f)(x)=g'(f(x))\cdot \gradient f(x)}$

\subsection{Kurvenapproximation}
  Punkte $P_1$ bis $P_n$ sind gegeben.
  \begin{enumerate}
    \item 
      Approximation durch eine Funktion: $y(x) = ax+b$\\
      $f(a;b) = \sum\limits_{i=1}^{n} (y(x_i)-y_i)^2 = \sum\limits_{i=1}^{n} (ax_i+b-y_i)^2$\\
    \item
      $f_a = \dprt{a} \sum\limits_{i=1}^{n} (ax_i+b-y_i)^2 = 2 \sum\limits_{i=1}^{n}x_i\cdot (ax_i+b-y_i)
      = 2 \sum\limits_{i=1}^{n}(ax_i^2+bx_i-x_iy_i)$ \\
      $f_b = \dprt{b} \sum\limits_{i=1}^{n} (ax_i+b-y_i)^2 = 2 \sum\limits_{i=1}^{n} (ax_i+b-y_i)$ \\
    \item
      $\gradient f = \left|\begin{matrix}
        f_a(a;b) \\
        f_b(a;b)
      \end{matrix} \right| = \left| \begin{matrix}
        2(a\bar{x^2} + b\bar{x} - \bar{xy}) \\
        2(a\bar{x} + b - \bar{y})
      \end{matrix} \right| = \left| \begin{matrix}
        a\bar{x^2} + b\bar{x} - \bar{xy} \\
        a\bar{x} + b - \bar{y}
      \end{matrix} \right| = \vec{0}$ \\
    \item
      Nach $a$ und $b$ auflösen und in die Funktion einsetzen
  \end{enumerate}
    
\subsection{numerische Verfahren zum Auffinden von stationären Punkten}
  \subsubsection{Newton Verfahren\formelbuch{29}}
    \begin{enumerate}
      \item
        Man startet mit einer möglichst guten Approximation $(x_0;y_0)$\\
      \item
        Man löst für $n = 0,1,2,\ldots$ das Gleichungssystem\\
        $\left|\begin{matrix}
          f_{xx}(x_n;y_n)\cdot\Delta x_n + f_{xy}(x_n;y_n)\cdot\Delta y_n = -f_x(x_n;y_n) \\
          f_{xy}(x_n;y_n)\cdot\Delta x_n + f_{yy}(x_n;y_n)\cdot\Delta y_n = -f_y(x_n;y_n)
        \end{matrix}\right|$ , und setze $(x_{n+1};y_{n+1}) := (x_n +\Delta x_n; y_n + \Delta y_n)$\\
      \item
        Den letzten berechneten Punkt $(x_{n+1};y_{n+1})$ akzeptiert man als Approximation für den stationären Punkt.
    \end{enumerate}
    
  \subsubsection{Gradientenverfahren \formelbuch{34}}
    Beim Gradientenverfahren startet man bei einer möglichst guten Approximation und folgt, dann immer der steilsten Steigung,
    bis genügend Konvergenz vorliegt.
    

\subsection{Extremalprobleme}
  \textbf{Zwei Variablen\formelbuch{25}}
  \begin{enumerate}
    \item 
      Randpunkte von $\mathbb{D}_f$
    \item 
      Punkte, in denen der Gradientenvektor $\gradient f$ nicht existiert. 
    \item 
      Punkte, in denen der Gradientenvektor $\gradient f=0$ ist.\\
      Hat die Funktion $f(x;y)$ an der Stelle $(x_0;y_0)$ einen verschwindenden Gradientenvektor $\gradient f = 0$ 
      und gilt für die Diskriminante $\Delta = f_{xx}    (x_0;y_0) \cdot f_{yy}(x_0;y_0) -  (f_{xy}(x_0;y_0))^2$
      \begin{itemize}
        \item 
          $\Delta > 0 $, so besitz $f(x_0;y_0)$ ein lokales Extremum. Im Fall $f_{xx}(x_0;y_0) < 0$ liegt ein lokales Maximum vor, 
          für $f_{xx}(x_0;y_0) > 0$ hingegen ein lokales Minimum.
        \item
          $\Delta < 0 $, so besitzt $f(x_0;y_0)$ in $(x_0;y_0)$ ein Sattelpunkt.
        \item 
          $\Delta = 0 $, so braucht es weitere Untersuchungen, um die Art der Stelle $(x_0;y_0)$ zu bestimmen können.
        \item 
          Wenn es kein lokales/relatives Minima/Maxima gibt, dann auch kein absolutes!
        \item
          Hat es ein lokales Maxima so gilt: $f(x;y) \leq M$ $\forall$ $(x;y) \in \mathbb{D}_f \Rightarrow M$ globales Maxima
      \end{itemize}
    \end{enumerate}
  
  \textbf{m Variablen\formelbuch{46}}
  Funktion $f(x_1^{(0)};\ldots;x_m^{(0)})$ gegeben.  \\
  \begin{itemize}
    \item 
      Schritte 1 - 3 sind gleich wie bei zwei Dimensionen. Kandidaten $(x_1^{(0)};\ldots;x_m^{(0)})$ bekommen.
    \item 
      Bestimmung der Art der Extremalstellen: Hessesche Matrix aufstellen\\
      $H(x_1^{(0)};\ldots;x_m^{(0)}) := \begin{pmatrix}
        f_{x_{1}x_{1}}(x_1^{(0)};\ldots;x_m^{(0)}) & \ldots & f_{x_{1}x_{m}}(x_1^{(0)};\ldots;x_m^{(0)})\\
        \vdots && \vdots\\
        f_{x_{m}x_{1}}(x_1^{(0)};\ldots;x_m^{(0)}) & \ldots & f_{x_{m}x_{m}}(x_1^{(0)};\ldots;x_m^{(0)})
      \end{pmatrix}$\\
      \begin{itemize}
        \item
          $m$ positive Eigenwerte $\lambda_i > 0$, so besitzt $f$ in $(x_1^{(0)};\ldots;x_m^{(0)})$ ein lokales Minimum,
        \item
          $m$ negative Eigenwerte $\lambda_i < 0$, so besitzt $f$ in $(x_1^{(0)};\ldots;x_m^{(0)})$ ein lokales Maximum,
        \item 
          positive und negative Eigenwerte, so besitzt $f$ in $(x_1^{(0)};\ldots;x_m^{(0)})$ einen Sattelpunkt.
        \item 
          Wenn $\lambda_i \leqq 0$ oder $\lambda_i \geqq 0$ sind weitere Untersuchungen nötig.
      \end{itemize}
  \end{itemize}

%\subsubsection{Extremstellen auf Randwerten \formelbuch{38}}
%\begin{tabular}{lll}
%	\begin{minipage}{3.5cm}
%		$\boxed{\gradient f - \lambda \gradient g = 0}$		
%	\end{minipage} &
%	\begin{minipage}{4.7cm}
%		$f: $ zu maximierende Funktion\\
%		$g: $ Funktion des Randes	
%    \end{minipage} &
%	\begin{minipage}{11cm}
%		1. Gleichung unter der Bedingung  $g=0$ auflösen\\
%		2. Lösungen durch einsetzen in $f$ auf Min o. Max untersuchen
%    \end{minipage}
%\end{tabular}

\subsection{Extremalprobleme mit Nebenbedingungen}
\begin{multicols}{2}
  \textbf{Zwei Variablen\formelbuch{42}}\\
  Gegeben: $f(x;y)$ unter der Nebenbedingung $n(x;y) = 0$\\
  So kommen folgende Punkte von $f$ in Frage:
  \begin{enumerate}
    \item 
      Randpunkte von $\mathbb{D}_f$ wenn sie die Nebenbedingungen  $n(x;y) = 0$ erfüllen und zu $\mathbb{D}_f$ gehören.
    \item 
      Punkte, in denen der Gradientenvektor $\gradient f $ und / oder $\gradient n$ nicht existieren, 
      und die Nebenbedingung
      $n(x;y) = 0$ erfüllen.
    \item 
      Lösungen des Gleichungsystems\\
      $ \begin{vmatrix}
        f_x(x;y) \cdot n_y(x;y) = f_y(x;y) \cdot n_x(x;y) \\
        n(x;y) = 0
     \end{vmatrix} $
   \item 
     Lösung in $f(x,y)$ einsetzen und untersuchen!
 \end{enumerate}
 \vfill
 
\columnbreak
  
 \textbf{m Variablen\formelbuch{49}}\\
 Falls eine Funktion $f(x_1;\ldots;x_m)$ unter den $k(<m)$ Nebenbedingungen $n_1(x_1;\ldots;x_m)=0, 
 \ldots, n_m(x_1;\ldots;x_m)=0$
 Maximal- oder Minimalstellen besitzt, kommen folgende Punkte in Frage:\\
 \begin{enumerate}
   \item
     Randpunkte von $\mathbb{D}_f$ falls sie zu $\mathbb{D}_f$ gehören und die Nebenbedingungen gleich 0 erfüllen.
   \item
     Punkte in denen (mindestens) einer der Gradientenvektoren $\gradient f, \gradient n_1, \ldots, \gradient n_k$ 
     nicht existieren
     oder in denen die $k$ Vektoren $\gradient n_1, \ldots, \gradient n_k$ linear abhängig sind und 
     die Nebenbedingungen erfüllen.
   \item
     Lösungen des $(m+k)\times(m+k)$ Gleichungssystems:\\
     $\begin{vmatrix}
       \gradient f(x_1;\ldots;x_m) & = & \lambda_1 \cdot \gradient n_1(x_1;\ldots;x_m)+ \ldots + \\
       & & +\lambda_k \cdot \gradient n_k(x_1;\ldots;x_m) \\
       n_1(x_1;\ldots;x_m) = 0 \\
       \vdots \\
       n_k(x_1;\ldots;x_m) = 0
     \end{vmatrix}$
   \item
     Lösungen in $f(x_1;\ldots;x_m)$ einsetzen.
 \end{enumerate}

\end{multicols}

\section{Mehrfache Integrale \formelbuch{50}}
\subsection{Oberflächenberechnung\formelbuch{73}}
\begin{tabular}{|p{5.5 cm}|p{10cm}|}
	\hline
	\begin{minipage}{5.5cm}
    	\vspace{0.1cm}
    	Oberflächenintegral\\
    	$f:G\rightarrow \mathbb{R}$
    	\vspace{0.1cm} 
    \end{minipage}&
	\begin{minipage}{10cm}
    	\vspace{0.1cm}
		$F = \iint\limits_G \sqrt{(\sprt{x})^2+(\sprt{y})^2+1}\cdot d\mu$
    	\vspace{0.1cm}
    \end{minipage}\\
	\hline
	\begin{minipage}{5.5cm}
    	\vspace{0.1cm}
    	Oberflächenintegral\\
    	$f:G\rightarrow \mathbb{R}^3$
    	\vspace{0.1cm} 
    \end{minipage}&
	\begin{minipage}{10cm}
    	\vspace{0.1cm}
		$F=\iint\limits_G\left|
		\begin{pmatrix}
	    	\fsprt{x}{f_1}\\
	    	\fsprt{x}{f_2}\\
	    	\fsprt{x}{f_3}\\              
	    \end{pmatrix}
		\times
		\begin{pmatrix}
	      	\fsprt{y}{f_1}\\
	    	\fsprt{y}{f_2}\\
	    	\fsprt{y}{f_3}\\   	
	    \end{pmatrix}\right|d\mu=
		\iint\limits_G |D_1f\times D_2f|\cdot d\mu$	
    	\vspace{0.1cm}
    \end{minipage}\\
	\hline
	\begin{minipage}{5.5cm}
  		Mantelfläche eines Rotationskörpers in Zylinderkoordinaten    
    \end{minipage}&
	\begin{minipage}{10cm}
    	\vspace{0.1cm}
		$M = 2\pi\int\limits_0^h\varrho(z)\sqrt{1+\varrho'(z)^2}\cdot dz$
    	\vspace{0.1cm}
    \end{minipage}\\
	\hline
	Polarkoordinaten &
	\begin{minipage}{10cm}
    	\vspace{0.1cm}
		$F=\iint\limits_G f(r,\varphi)\cdot \textcolor{blue}{r} \cdot dr d\varphi $
    	\vspace{0.1cm}
    \end{minipage}\\
	\hline
\end{tabular}

\subsection{Bereichsintegral\formelbuch{56}}
\begin{multicols}{2}
  $\boxed{\int\limits_{x_{min}}^{x_{max}}\int\limits_{y_{min}(x)}^{y_{max}(x)}f(x;y) dy dx =
  \int\limits_{y_{min}}^{y_{max}}\int\limits_{x_{min}(y)}^{x_{max}(y)}f(x;y) dx dy }$
  
\columnbreak
  $\boxed{\int\limits_{x_{1_{min}}}^{x_{1_{max}}}\int\limits_{x_{2_{min}}(x_1)}^{x_{2_{max}}(x_1)}\ldots 
  \int\limits_{x_{m_{min}}(x_1;\ldots;x_m)}^{x_{m_{max}}(x_1;\ldots;x_m)}f(x_1;\ldots;x_m)dx_m \ldots dx_2 dx_1}$
  
\end{multicols}


\subsection{Volumenberechnung\formelbuch{55/56/66}}
  \begin{multicols}{2}
    $\boxed{V = \int\limits_{x_{min}}^{x_{max}}\int\limits_{y_{min}(x)}^{y_{max}(x)}f(x;y) dy dx}$ \\
    
  \columnbreak
    $\boxed{V = \int\limits_V dV = \int\limits_{x_{1_{min}}}^{x_{1_{max}}}\int\limits_{x_{2_{min}}(x_1)}^{x_{2_{max}}(x_1)}
    \int\limits_{x_{3_{min}}(x_1;x_2)}^{x_{3_{max}}(x_1;x_2)}1 dx_3 dx_2 dx_1}$
  \end{multicols}


\subsection{Schwerpunkt}
  \begin{multicols}{2}
    \subsubsection{einer dünnen ebener Platte\formelbuch{59}}
      \textbf{inhomogen}:\\
        $x_s=\frac{1}{M} \int\limits_F x \cdot \varrho(x,y,z)\cdot dF$\\
        $y_s=\frac{1}{M} \int\limits_F y \cdot \varrho(x,y,z)\cdot dF$\\
      \textbf{homogen}:\\
        mit $\varrho$ konstant und Objekt nur Platte ($z$ entfaellt):\\
        $x_s=\frac{1}{F} \int\limits_F x\cdot dF = \frac{1}{F} \int\limits_{x_{min}}^{x_{max}}
        x(y_{max}(x)-y_{min}(x))\cdot dx$ \\
        $y_s=\frac{1}{F} \int\limits_F y\cdot dF = \frac{1}{2F} \int\limits_{x_{min}}^{x_{max}}
        (y_{max}^2(x)-y_{min}^2(x))\cdot dx$ \\
    \columnbreak
    
    \subsubsection{von Körpern\formelbuch{67}}
      \textbf{inhomogen}: \\
        $x_s = \frac{1}{M}\int\limits_V x \cdot \varrho(x;y;z) dV \qquad$ analog für $y_s$ und $z_s$\\
      \textbf{homogen}: \\
        $x_s = \frac{1}{V}\int\limits_V x dV \qquad$ analog für $y_s$ und $z_s$\\
      
      $M$ = totale Masse\\
      $F$ = Fläche
  \end{multicols}
   
\subsection{Trägheitsmoment\formelbuch{67}}
  $J = \int\limits_V \varrho(x;y;z) \cdot [r(x;y;z)]^2 dV$ \\
  Für homogene Körper gilt: $J = \varrho \int\limits_V \underbrace{(x^2+y^2)}_{r^2}dV$

\newpage

\subsection{Koordinatentransformation\formelbuch{62/64/67}}
  $\int\limits_B f(x;y) dF = \int\limits_B \underbrace{f(x(u;v); y(x;v))}_{\tilde{f}(u;v)} \cdot Korrekturfaktor(u;v)   
  \tilde{dF}$
    
  $\int\limits_B f(x;y) dF = \int\limits_B \underbrace{f(x(u;v); y(x;v))}_{\tilde{f}(u;v)} \cdot 
  \left|\left|\dfrac{\partial (x;y)}{\partial (u;v)}\right|\right| \tilde{dF}$
    
  $\int\limits_B f(x_1;\ldots;x_m) dB = \int\limits_{\tilde{B}} 
  \underbrace{f(x_1(u_1;\ldots;u_m);\ldots; x_m(u_1;\ldots;u_m))}_{\tilde{f}(u_1;\ldots;u_m)} \cdot 
  \left|\left|\dfrac{\partial (x_1;\ldots;x_m)}{\partial (u_1;\ldots;u_m)}\right|\right| \tilde{dB}$ \\
   
  Der Korrekturfaktor ist der Betrag der Jacobideterminante $|detDf|$

 
  \subsubsection{Jacobimatrix (Funktionalmatrix)\formelbuch{63/67}}
    \begin{tabular}{|l|l|l|}
      \hline
        Jacobimatrix ($Df$): &
        $\dfrac{\partial(x,y)}{\partial(u,v)} = \begin{pmatrix}
          x_u & x_v \\
          y_u & y_v
        \end{pmatrix}$ &

        $\dfrac{\partial (x_1;\ldots;x_m)}{\partial (u_1;\ldots;u_m)} = \begin{pmatrix}
          \frac{\partial x_1}{\partial u_1} & 
          \ldots & 
          \frac{\partial x_1}{\partial u_m} \\
    
          \vdots & & \vdots \\
    
          \frac{\partial x_m}{\partial u_1} & 
          \ldots & 
          \frac{\partial x_m}{\partial u_m}
        \end{pmatrix} $ \\
      \hline
        Jacobideterminante ($detDf$): &
        $\left|\dfrac{\partial(x,y)}{\partial(u,v)}\right| = \begin{vmatrix}
          x_u & x_v \\
          y_u & y_v
        \end{vmatrix}$ &
      
        $\left|\dfrac{\partial (x_1;\ldots;x_m)}{\partial (u_1;\ldots;u_m)}\right| = \begin{vmatrix}
          \frac{\partial x_1}{\partial u_1} & 
          \ldots & 
          \frac{\partial x_1}{\partial u_m} \\
    
          \vdots & & \vdots \\
    
          \frac{\partial x_m}{\partial u_1} & 
          \ldots & 
          \frac{\partial x_m}{\partial u_m}
        \end{vmatrix} $\\
      \hline
    \end{tabular}
  
 
    
    
  \subsection{Koordinatensysteme}
\begin{tabular}{|p{2.5cm}||p{3cm}|p{4.2cm}|p{7.5cm}|}
	\hline
	$f(x,y,(z))\quad\rightarrow$ &
	\begin{minipage}{2.4cm}
    	\vspace{0.1cm}
		$f(r,\varphi)\quad$\textbf{Polar} 
    	\vspace{0.1cm}    	
    \end{minipage}& 
	$f(r,\varphi,z)\quad$ \textbf{Zylinder} &
	$f(r,\varphi,\vartheta)\quad$\textbf{Kugel}\\
	\hline
	\hline
	Bilder &
	\begin{minipage}{3cm}
    	\includegraphics[width=3cm]{../FuVar/images/Ebene_polarkoordinaten.png}
    \end{minipage}&
	\begin{minipage}{3cm}
    	\includegraphics[width=4.2cm]{../FuVar/images/Zylinderkoordinaten.png}
    \end{minipage}&
	\begin{minipage}{3cm}
    	\includegraphics[width=3cm]{../FuVar/images/Kugelkoordinaten2.png}
    \end{minipage}\\
	\hline
	Umrechnungs- formeln &
	\begin{minipage}{3cm}
    \vspace{0.1cm}
		$x=r\cos\varphi$\\
		$y=r\sin\varphi$    
    \vspace{0.1cm}
    \end{minipage}&	
	\begin{minipage}{4.2cm}
    \vspace{0.1cm}
    	$x=r\cos\varphi$\\
    	$y=r\sin\varphi$\\
    	$z=z$
    \vspace{0.1cm}
    \end{minipage}&	
	\begin{minipage}{7.5cm}
    \vspace{0.1cm}
    	$x=r\cos\vartheta\cos\varphi \quad\quad r \geq 0,$\\
    	$y=r\sin\vartheta\cos\varphi \quad\quad  0\leq\vartheta\leq 2\pi,$\\
    	$z=r\sin\vartheta \quad\quad\quad
    	-\frac{\pi}{2}\leq\varphi\leq\frac{\pi}{2}$
    \vspace{0.1cm}
    \end{minipage}\\
	\hline
	Jacobi-Matrix $Df$ &
	\begin{minipage}{3cm}
    \vspace{0.1cm}
		$\begin{pmatrix}
         	\cos\varphi & -r\sin\varphi\\
         	\sin\varphi & r\cos\varphi
         \end{pmatrix}$
    \vspace{0.1cm}
    \end{minipage}&	
	\begin{minipage}{4.2cm}
    \vspace{0.1cm}
		$\begin{pmatrix}
         	\cos\varphi & -r\sin\varphi & 0\\
         	\sin\varphi & r\cos\varphi & 0\\
         	0 & 0 & 1
         \end{pmatrix}$
    \vspace{0.1cm}
    \end{minipage}&	
	\begin{minipage}{7.5cm}
    \vspace{0.1cm}
		$\begin{pmatrix}
         	\cos\vartheta\cos\varphi & -r\sin\vartheta\cos\varphi &
         	-r\cos\vartheta\sin\varphi\\
         	\sin\vartheta\cos\varphi & r\cos\vartheta\cos\varphi &
         	-r\sin\vartheta\sin\varphi\\
         	\sin\vartheta & 0 & r\cos\vartheta
         \end{pmatrix}$
    \vspace{0.1cm}
    \end{minipage}\\
	\hline
	\begin{minipage}{2.5cm}
    	\vspace{0.1cm}
  		$detDf$   
  		\vspace{0.1cm}
    \end{minipage}&
		$r$ &
		$r$ &
		$r^2\cos\vartheta$\\
	\hline	
\end{tabular}
    



\newpage
\section{Kurven ($\mathbb{R} \rightarrow \mathbb{R}^n$)
\formelbuch{41}}
\subsection{Parametertransformation\formelbuch{42}}
Tangentenrichtung ist unabhängig von der Parametrisierung der Kurve. Die Länge
jedoch nicht. Ist $g$ eine Kurve, beschreibt $f\circ g$ die gleiche Kurve mit
einer anderen Parametrisierung. Der Tangentialvektor ist somit:\\
$$\frac{d(f\circ g)}{dt}(t_0)=\frac{df}{dt}(g(t_0))\frac{dg}{dt}(t_0)=
\frac{df}{dt}(g(t_0))g'(t_0)$$\\
die Länge des Vektors wird also mit dem Faktor $f'(t_0)$ gestreckt oder
gestaucht.

\subsection{Kurvenlänge\formelbuch{42}}
\begin{tabular}{|l||l|l|}
\hline
& \textbf{Formel} & \textbf{Bedingung(en)}\\
\hline
\hline
\textbf{Länge von $a$ nach $b$} &
	\begin{minipage}{5cm}
    	\vspace{0.1cm}
		$L=\int\limits_a^b|{f'(\tau)}|d\tau$ 
		\vspace{0.1cm}
    \end{minipage}&
rektifizierbar $\quad\Rightarrow\quad f'(t)$ integrierbar\\
\hline
\textbf{Länge in Abhängigkeit der Zeit} &
	\begin{minipage}{5cm}
    	\vspace{0.1cm}
		$s(t)=\int\limits_a^t|{f'(\tau)}|d\tau$ 
		\vspace{0.1cm}
    \end{minipage}&
\begin{minipage}{7cm}
	rektifizierbar $\quad\Rightarrow\quad f'(t)$ integrierbar\\
	regulär $\qquad\quad\;\;\Rightarrow\quad f'(t)\neq 0$
\end{minipage}\\
\hline
\end{tabular}\vspace{0.5cm}\\

\textbf{Flächeninhalt einer geschlossenen ebenen Kurve}\\
Eine kreuzungsfreie, geschlossene Kurve $\begin{pmatrix} x\\ y \end{pmatrix} = \begin{pmatrix} x(t)\\ y(t) \end{pmatrix}, t \in (r;s)$ hat einen Flächeninhalt von\\
$\boxed{F =  \left | \int_r^s y(t) \cdot x^{'}(t) dt \right | = \left | \int_r^s
x(t) \cdot y^{'}(t) dt \right |}$\\\\
Hat die Kurve einen Kreuzungspunkt:\\ 
$x_1(t_1) = x_2(t_2)$\\
$y_1(t_1) = y_2(t_2)$\\\\
Der Zeitpunkt t ist bei $x_1$ und $x_2$ \textbf{NICHT} gleich!\\
So kann man herausfinden, innerhalb welchen Zeitpunkten die Schleife durchgangen
wird!\\

\textbf{Kurve in Kurvenlängenparameter transformieren}
\begin{enumerate}
  \item Prüfen ob $f'(t)\neq 0$ \& integrierbar ist, falls nicht kann die Kurve
  nicht transformiert werden.
  \item $s(t)$ berechnen.
  \item Umkehrfunktion $s(t)^{-1}$ berechnen.
  \item $t(s)$ in $f$ einsetzen $\Rightarrow f(t(s))= f\circ s(t)^{-1}$
\end{enumerate}
\vspace{0.5cm}
\textbf{Eigenschaften für Kurvenlängenparameter}\\
Eine Kurve besitzt Kurvenlängenparameter wenn folgendes gilt:\\
$$\left|\frac{df}{dt}(t)\right|=1$$

\subsection{Krümmung \& Krümmungskreisradius\formelbuch{45}}
\begin{tabular}{|l|l|l|}
	\hline
	& mit Kurvenlägenparameter & ohne Kurvenlängenparameter\\
	\hline
	Krümmung &
	$\kappa(s)=|f''(s)|$ &
	\begin{minipage}{5cm}
    	\vspace{0.1cm}
		$\kappa(t)=\dfrac{|f'(t) \times f''(t)|}{|f'(t)|^3}$\\ 
		\vspace{0.1cm}
    \end{minipage}\\	
	\hline
	Krümmungskreisradius &
	$r(s)=\dfrac{1}{\kappa(s)}=\dfrac{1}{|f''(s)|}$ &
	\begin{minipage}{5cm}
    	\vspace{0.1cm}
		$r(t)=\dfrac{1}{\kappa(t)}=\dfrac{|f'(t)|^3}{|f'(t) \times
		f''(t)|}$    
		\vspace{0.1cm}
    \end{minipage}\\
	\hline
\end{tabular}
\newpage


%\section{Abbildungen ($\mathbb{R}^n \rightarrow \mathbb{R}^m$)
\formelbuch{55}}
\subsection{Ableitungen\formelbuch{55}}
$Df(x) = \begin{pmatrix}
    	\fprt{f_1}{x_1}(x) & \ldots & \fprt{f_1}{x_n}(x)\\
    	\vdots && \vdots\\
    	\fprt{f_m}{x_1}(x) & \ldots & \fprt{f_m}{x_n}(x)
	\end{pmatrix}=\dfrac{\partial(f_1,\ldots,f_m)}{\partial(x_1,\ldots,x_n)}
	\quad \Rightarrow \quad$ Jacobi-Matrix (Ableitung in alle Koordinatenrichungen)
\subsection{Kettenregel\formelbuch{56}}
\begin{minipage}{3cm}
	$f: \mathbb{R}^n \rightarrow \mathbb{R}^m$\\
	$g: \mathbb{R}^m \rightarrow \mathbb{R}^r$
\end{minipage}
\begin{minipage}{7cm}
	$\boxed{h=D(g\circ f)(x)=Dg(f(x_0))\cdot Df(x_0)}$
\end{minipage}
\begin{minipage}{4cm}
	$h: \mathbb{R}^n \rightarrow \mathbb{R}^r$
\end{minipage}

\subsection{Volumenänderung\formelbuch{58}}
$\Delta V = \begin{vmatrix}
    	\fprt{f_1}{x_1}(x) & \ldots & \fprt{f_1}{x_n}(x)\\
    	\vdots && \vdots\\
    	\fprt{f_m}{x_1}(x) & \ldots & \fprt{f_m}{x_n}(x)
		\end{vmatrix}=\left|\dfrac{\partial(f_1,\ldots,f_m)}
		{\partial(x_1,\ldots,x_n)}\right|= det Df$

\subsection{Newtonverfahren\formelbuch{60}}
Mit dem Newtonverfahren können rekursiv Nullstellen approximiert werden.\\\\
$\boxed{\vec{x}_{neu}=\vec{x}_{alt}-(Df(\vec{x}_{alt}))^{-1}\cdot
(f(\vec{x}_{alt})-\vec{y}_{Ziel})}\qquad$ allgemein\\
$\boxed{x_{neu}=x_{alt}-(f'(x_{alt}))^{-1}\cdot
(f(x_{alt})-y_{Ziel})}\qquad$ (1- Dimensional) wird
rekursiv ausgeführt, bis nötige Genauigkeit da ist.

\newpage
\section{Vektorfelder\formelbuch{82}}
  Ein Vektorfeld ist eine Abbildung $\vec{v}:\mathbb{R}^n\rightarrow
  \mathbb{R}^n$, welche jedem Punkt des Raumes einen Vektor anheftet.
  
\subsection{Wegintegral\formelbuch{83}}
  Ist $\vec{F}$ ein Vektorfeld und $C:\vec{c}=\vec{c}(t),r\leq t \leq s$ eine
  Kurve, dann wird das Wegintegral wie folgt berechnet:\\\\
  $\boxed{\int\limits_C\vec{F}\bullet d\vec{s}=
  \int\limits_r^s\vec{F}(\vec{c}(t))\bullet \dot{\vec{c}}(t)\cdot dt}$
  z.B: $\vec{F}=\begin{pmatrix}
    -y \\
    x \\
    z
  \end{pmatrix}$ und $\vec c = \begin{pmatrix}
    1t \\
    3t \\
    5t 
  \end{pmatrix} \Rightarrow \int\begin{pmatrix}
    -3t\\
    1t \\
    5t
  \end{pmatrix} \bullet \begin{pmatrix}
    1\\
    3\\
    5
  \end{pmatrix} dt$
  \\\\
  Tipp: Wenn c bspw. eine gerade Linie vom Punkt (-1;0) bis (1;0) ist, dann ist:\\
  $\vec{c}(t)=
  \begin{pmatrix}
    x(t)\\
    y(t)\\
  \end{pmatrix}
  =
  \begin{pmatrix}
    t\\
    0\\
  \end{pmatrix}$
  mit $-1\leq t \leq 1$

  \textbf{Eigenschaften des Wegintegrals}
  \begin{itemize}
    \item Ist $C$ eine geschlossene Kurve, also $\vec{c}(r)=\vec{c}(s)$, spricht
    man von einem \textbf{Umlaufintegral} und verwendet folgende Definition:
    $\oint\limits_C \vec{F}\bullet d\vec{s}$
    \item Ist $C=C_1+C_2$ gilt folgendes:$\quad\int\limits_C \vec{F}\bullet
    d\vec{s}=\int\limits_{C_1} \vec{F}\bullet d\vec{s}+\int\limits_{C_2}
    \vec{F}\bullet d\vec{s}$
    \item Wird $C$ in Gegenrichtung durchlaufen gilt folgendes:
    $\quad\int\limits_C \vec{F}\bullet
    d\vec{s}=-\int\limits_{-C} \vec{F}\bullet d\vec{s}$
  \end{itemize}

\subsection{Konservative Felder / Potentialfelder\formelbuch{83}}
  Ein Vektorfeld $\vec{F}$ heisst \textit{konservativ}, wenn das Wegintegral
  unabhängig vom gewählten Weg ist.\\\\
  $\Rightarrow\qquad \boxed{\oint_C \vec{F}\bullet d\vec{s}=0}\qquad\boxed{
  \int\limits_{Weg_1\, A\rightarrow B} \vec{F}\bullet d\vec{s}=\int\limits_{Weg_2\,
  A\leadsto B} \vec{F}\bullet d\vec{s}=\int\limits_A^B \vec{F}\bullet
  d\vec{s}}\qquad\boxed{\rotation (\vec{F})=0}$\\\\\\	
  \textbf{Es gelten folgende Sätze:}
  Für ein Vektorfeld $\vec{F}$ im zwei- oder dreidimensionalen Koordinatensystem
  gilt in einem Bereich ohne "`durchgehende Löcher"', d.h. in dem sich jede
  geschlossene Kurve auf einen Punkt zusammenziehen lässt:
  \begin{itemize}
    \item $\vec{F}$ ist konservativ, d.h die Kurvenintegrale wegunabhängig.
    \item $\vec{F}$ ist ein Gradientenfeld, d.h. es gibt ein Potential
    $\Phi(x_1,\ldots,x_n)$ mit $\vec{F}=\gradient \Phi(\vec{x}(t))$.
    \item $\vec{F}$ erfüllt die sogenannte(n)
    \textbf{Integrabilitätsbedingung(en)}(dient zur Überprüfung ob es konservativ ist), d.h. im
    \begin{itemize}
      \item zweidimensionalen Fall die Gleichung $\frac{\partial F_1}{\partial
      x_2}=\frac{\partial F_2}{\partial x_1}$,
      \item dreidimensionalen Fall die Gleichungen $\frac{\partial F_1}{\partial
      x_2}=\frac{\partial F_2}{\partial x_1}$, $\frac{\partial F_2}{\partial
      x_3}=\frac{\partial F_3}{\partial x_2}$, $\frac{\partial F_3}{\partial
      x_1}=\frac{\partial F_1}{\partial x_3}$
    \end{itemize}
  \end{itemize}
  \textbf{und}
  \begin{itemize}
    \item Ist $\vec{F}=\gradient \Phi(\vec{x}(t))$, dann ist $\vec{F}$  
    \textit{konservativ} und es gilt: $\quad\int\limits_r^s \vec{F}\bullet \dot{\vec{x}}
    dt=\Phi(\vec{x}(s))-\Phi(\vec{x}(r))$
    \item Ist $\vec{F}$ \textit{konservativ}, dann gibt es eine Funktion
    $\Phi(x_1,\ldots,x_n)$ so, dass $\vec{F}=\gradient \Phi(\vec{x}(t))=
    \begin{pmatrix}
      \Phi_{x_1}\\
      \vdots\\ 
      \Phi_{x_n}
    \end{pmatrix}$.

  \end{itemize}


\subsection{Rotation}
  Interpretiert man dieses Feld als Strömungsfeld, so gibt die Rotation für jeden
  Ort an, wie schnell und um welche Achse ein mitschwimmender Körper rotieren
  würde. Ein Vektorfeld, dessen Rotation überall null ist, nennt man
  \textit{wirbelfrei}.\\\\
  \begin{minipage}{8cm}
    	\textbf{2-Dimensional}\\\\
    	$\boxed{\rotation(u,v)=\fprt{v}{x}-\fprt{u}{y}}$
  \end{minipage}
  \begin{minipage}{8cm}
	  \textbf{3-Dimensional}\\\\
    	$\boxed{\rotation(\vec F)=\bigtriangledown\times\vec{F} = 
    	\begin{pmatrix}
    	  (F_3)_y - (F_2)_z\\
    	  (F_1)_z - (F_3)_x\\
    	  (F_2)_x - (F_1)_y
    	\end{pmatrix}}$
  \end{minipage} \\

\subsection{Divergenz}
  Interpretiert man dieses Feld als Strömungsfeld, so gibt die Divergenz für jede
  Stelle die Tendenz an, ob ein Teilchen in der Nähe zu diesem Punkt hin- bzw.
  von diesem Punkt wegfließt. Es sagt damit aus, ob und wo das Vektorfeld Quellen
  (Divergenz größer als Null) oder Senken (Divergenz kleiner als Null) hat. Ist
  die Divergenz überall gleich Null, so bezeichnet man das Feld als
  \textit{quellenfrei}.\\\\
  $\boxed{\divergenz \vec{F}=\bigtriangledown\circ \vec{F}=
	\begin{pmatrix}
    	\frac{\partial}{\partial x_1 }\\
    	\vdots\\
    	\frac{\partial}{\partial x_n}
    \end{pmatrix}\cdot
	\begin{pmatrix}
    	F_1\\
    	\vdots\\
    	F_n
  \end{pmatrix}=
  \sum\limits_{i=1}^n \fprt{F_i}{x_i}}$

\subsection{Fluss}
  Der Fluss des Vektorfeldes $\vec{F}$ durch das Flächenstück G mit
  Parametrisierung $f:U\rightarrow\mathbb{R}^3$ ist:\\\\
  $\boxed{\lim\limits_{\Delta A \to 0}\sum \vec F \bullet \Delta\vec A = \int\limits_A \vec F \cdot \vec{\Delta A}}$

\subsection{Integralsätze}
	\begin{tabular}{|p{0.4cm}||p{4cm}|p{5.7cm}|p{7cm}|}
	\hline
	& \textbf{Satz} & \textbf{Formel} & \textbf{Beschreibung}\\
	\hline
	\hline
	\begin{sideways}2-D \qquad \end{sideways} &
  	Die Greensche Formel &
	\begin{minipage}{6.7cm}
	    \vspace{0.1cm}
		$\int\limits_A \frac{\partial F_2}{\partial x} - \frac{\partial F_1}{\partial y} dA = \oint\limits_{\partial A}\vec{F}\bullet d\vec{s}$	 		    
	    \vspace{0.1cm}   
    \end{minipage}&
	\begin{minipage}{7cm}
	    \vspace{0.1cm}
		Die Greensche Formel beschreibt den Zusammenhang zwischen einem
		\textbf{Wegintegral} und einem \textbf{Oberflächenintegral}.
	    \vspace{0.1cm}   
    \end{minipage}\\
	\hline
	\begin{sideways}2-D \qquad \end{sideways} &
	Der Satz von Stokes &
	\begin{minipage}{6.7cm}
    	\vspace{0.1cm}
		$\int\limits_A \rotation \vec{F}\cdot d\vec{A}  = \oint\limits_{\partial A}\vec{F}\bullet d\vec{s}$		 
		\vspace{0.1cm} 
    \end{minipage}&
	\begin{minipage}{7cm}
    	\vspace{0.1cm}
		Der Satz von Stokes definiert einen Zusammenhang zwischen einem
		\textbf{Wegintegral} und einem \textbf{Flussintegral}.	    
	    \vspace{0.1cm}   
    \end{minipage}\\
	\hline
	\begin{sideways}3-D \qquad \end{sideways} &
	Der Satz von Gauss  &
	\begin{minipage}{6.7cm}
	    \vspace{0.1cm}
		$\int\limits_V \divergenz \vec F dV = \oint\limits_{\partial V} \vec F \bullet \vec{dA}$			    
	    \vspace{0.1cm}   
    \end{minipage}&
	\begin{minipage}{7cm}
    	\vspace{0.1cm}
		Der Satz von Gauss definiert einen Zusammenhang zwischen einem
		\textbf{Flussintegral} und einem \textbf{Volumenintegral}.	    
	    \vspace{0.1cm}    
    \end{minipage}\\
	\hline	
\end{tabular}

\section{Diverses}

\subsection{Rechenregeln für div, rot, grad, $\Delta$}
\begin{minipage}{9cm}
Sei $D \in \mathbb{R}^3, f \in C^2(D,\mathbb{R}), \vec{v} \in C^2(D,\mathbb{R}^3)$. Dann gilt:
\begin{itemize}
	\item $\rotation(\gradient f) = 0 \qquad $ ``Gradientenfeld ist wirbelfrei''
	\item $\divergenz(\rotation \vec{v}) = 0 \qquad $ ``Feld der Rotation ist quellfrei''
	\item $\divergenz(\gradient f) = \Delta f$
	\item $\divergenz(f\vec{v}) = (\gradient f) \cdot \vec{v} + f \divergenz \vec{v}$
	\item $\rotation(f\vec{v}) = (\gradient f) \times \vec{v} + f \rotation \vec{v}$
	\item $\rotation(\rotation \vec{v}) = \gradient(\divergenz \vec{v}) - \Delta \vec{v}$
\end{itemize}
(Laplace–Operator kompomentenweise anwenden!) .
\end{minipage}
\hspace{1cm}
\begin{minipage}[b]{6cm}
\textbf{Laplace-Operator($\Delta$):}\\
$\Delta=\vec\nabla^2= \sum_{k=1}^n {\partial^2\over \partial x_k^2}$\\\\
\textbf{ Nabla-Operator($\nabla$):}\\
$\vec\nabla = \left (\frac\partial{\partial x_1},\ldots, \frac\partial{\partial
x_n}\right) $
\end{minipage}

\subsection{Eigenwerte}
  Die Eigenwerte $\lambda$ erhält man folgendermassen ($I$ ist die Einheitsmatrix):\\
  $|\lambda I - A| = 0 \qquad \Rightarrow$ nach $\lambda$ auflösen

%%%%%%%%%%%%%%%%%%%%%%%%%%%%%%%%%%%%%%%%%%%%%%%%%%%
% Idiotenseite
%%%%%%%%%%%%%%%%%%%%%%%%%%%%%%%%%%%%%%%%%%%%%%%%%%%
\newpage
\section{Idiotenseite}
\input{idiotenseite/trigo/subsections/Winkelargumente}
\input{idiotenseite/trigo/subsections/Periodizitaet}
\input{idiotenseite/trigo/subsections/Quadrantenbeziehungen}
\begin{multicols}{2}
  \input{idiotenseite/trigo/subsections/Additionstheoreme}
  \columnbreak
  
  \input{idiotenseite/trigo/subsections/DoppelHalbwinkel}
\end{multicols}
\begin{multicols}{2}
  \input{idiotenseite/trigo/subsections/Produkte}
  \columnbreak
  
  \input{idiotenseite/trigo/subsections/SummeDifferenzen}
\end{multicols}
\input{idiotenseite/diverses/subsections/unbestimmteIntegrale}
\input{idiotenseite/diverses/subsections/Ableitungen}


%\newpage
%\section{Übungsverzeichnis}
\renewcommand{\arraystretch}{1.05}
\begin{tabular}{|l|l|l|l|}
\hline
\textbf{Thema} & \textbf{Übung} & \textbf{Probeprüfung} & \textbf{Modulprüfung 07} \\ \hline
Achsabschnitte der Tangentialebene & 2.2 &  &  \\ \hline
Approximation mit Tangentialebene & 2.3 &  &  \\ \hline
Bogenlängenparameter &  & 4 &  \\ \hline
Definitionsbereich & 1.2 &  &  \\ \hline
Differenzialgleichung &  &  & 3 \\ \hline
Differenzierbarkeit &  & 2 &  \\ \hline
Divergenz & 13.3 / 13.4 & 8 &  \\ \hline
Elektromagnetisches Feld & 13.4 &  &  \\ \hline
Extremalprobleme & 4.3 &  &  \\ \hline
Extremalstellen auf Randwerten & 5.1 / 5.2 &  &  \\ \hline
Falllinien & 4.2 & 9 &  \\ \hline
Feldstärke in Abhängigkeit vom Radius & 13.2 &  &  \\ \hline
Flächenänderung & 7.3 &  &  \\ \hline
Flächenberechnung mit der Greenschen Formel & 11.4 &  &  \\ \hline
Flugzeug am exponentiellen Berg & 3.5 &  &  \\ \hline
Fluss & 12.2 / 12.3 / 13.1 &  &  \\ \hline
Gauss (Satz) & 13.1 &  &  \\ \hline
Gebiete & 8.1 & 8 &  \\ \hline
Gradient & 3.1 &  &  \\ \hline
Greensche Formel & 11.2 / 11.3 &  & 8 \\ \hline
Grenzwerte eines Gradienten & 3.6 &  &  \\ \hline
Implizite Differentiation & 4.1 &  &  \\ \hline
Insektenmännchen & 3.4 &  &  \\ \hline
Jacobimatrix & 7.2 / 7.3 / 7.4 &  &  \\ \hline
Krümmung / Krümmungsradius & 5.3 & 4 &  \\ \hline
Krümmungmittelpunkt & 5.4 &  &  \\ \hline
Maximieren einer Funktion & 5.1 / 5.2 & 5 & 5 \\ \hline
Maximum & 4.3 & 1 &  \\ \hline
Mehrfachintegrale & 8.2 / 8.3 / 8.4 & 3 & 1 / 4 / 7 \\ \hline
Minimum & 4.3 & 1 &  \\ \hline
Newtonverfahren & 7.1 / 7.2 / 7.5 &  &  \\ \hline
Niveaulinien & 3.3 &  &  \\ \hline
Oberflächenberechnung & 9.3 / 10.3 &  &  \\ \hline
Parameterdarstellung &  & 3 & 1 \\ \hline
Partielle Ableitung & 2.1 &  &  \\ \hline
Pfadintegral, Wegintegral & 6 / 11.2 & 7 & 8 \\ \hline
Potentialfeld & 6 / 11.1 &  & 6 \\ \hline
Richtungsableitung &  & 2 &  \\ \hline
Rotation & 11.1 &  &  \\ \hline
Rotationskörper & 9.1 / 9.2 &  &  \\ \hline
Sattelpunkt & 4.3 & 1 &  \\ \hline
Schnittkurve von zwei Funktionen & 2.4 &  &  \\ \hline
Schwerpunkt berechnen & 10.1 / 10.2 / 10.3 / 10.4 &  &  \\ \hline
Steigung impliziter Funktionen & 4.1 &  &  \\ \hline
Stokes (Satz) & 12.1 &  & 6 \\ \hline
Tangentialebene & 2.2 / 2.3 &  &  \\ \hline
Trägheitsmoment & 10.2 & 6 &  \\ \hline
Volumenänderung & 7.4 &  &  \\ \hline
Volumenberechnung & 8.2 / 8.3 / 8.4 / 9.1 / 9.2 & 3 & 1 / 4 / 7 \\ \hline
Wertebereich & 1.2 &  &  \\ \hline
Winkel zwischen zwei Funktionen & 2.4 &  &  \\ \hline
\end{tabular}

\renewcommand{\arraystretch}{1}

\end{document}
